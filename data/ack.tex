
%%% Local Variables:
%%% mode: latex
%%% TeX-master: "../main"
%%% End:

\begin{ack}
  行文至此,心中万分感慨。回顾博士阶段的五年学习生活,经历过认可时的激动和喜悦,也经历过失意时的苦涩和迷茫,少了一份功利,多了一份专注,这些带给我的不仅是科学思维上的磨练,更是全面素质的提升。在此,谨向多年来一直给予我知道、支持帮助的老师、同学和亲人们表示最衷心的感谢,是你们的陪伴和无私帮助激励我不断前行!
  
  衷心感谢我的导师沈荣骏院士。沈院士虽然不在长沙亲身指导,但定期他总要从北京赶来听取学生在课题方面的进展汇报,并给出很多建设性的意见和建议,来长沙出差之际,总是挤出时间和学生交流沟通。
  
  衷心感谢我的博士导师唐朝京教授!我从 2013 年受教于唐老师就读博士研究
  生至今,他严肃认真的治学态度、深厚的学术功底、严密的思维、渊博的知识和
  分析问题、提炼问题的能力,使我在学业上受益匪浅。我在博士论文选题、研究
  和论文写作的整个过程,都是在唐老师的悉心指导和严格要求下完成的。在几年
  的学习过程中,不仅使我的学术水平得到了很大的提高,在学术道德方面,我也
  有了更加深刻的认识,对自己的要求也更为严格。唐老师极富责任感、对待科研
  非常严谨,这种品质深深地影响着我,促使我形成严谨求实的学术作风,这将成
  为我人生的一大笔财富。唐老师还对我的论文写作方面进行了细心的指导,无私
  地传授给我论文写作的大量规律和经验。唐老师对工作认真勤恳的态度以及在学
  术上不断进取的精神永远是我今后工作学习的榜样!在此谨向他表示最衷心的感
  谢和最诚挚的敬意!
  
  衷心感谢张权副教授!是您带领我走进了信息安全研究的科研殿堂。感谢王
  剑教授、刘俭老师、张琛老师、张磊副教授、冯超老师、李瑞林老师等诸位老师对我的指导
  和关心。我在教研室学习期间,他们为我创造了良好的工作环境,他们在学术上
  的执着追求以及一丝不苟的工作作风也使我受
  
  感谢李孟君师兄、吴波师兄、解纬师兄、张博师兄、王少磊师兄、帅博师兄,王强师兄,感谢毕兴、刘毅、张斌、陈夏阳、叶嘉曦、张兴、冯冈夫、黄安琪、在同一个实验室中工作使得我们有了更多的讨论,在相互的交流和学习中我受益良多。
  
  深深的感谢我的家人,没有你们的支持,就没有今天的我!
  
  谨以此文献给所有关心我支持我的亲人和朋友们!

\end{ack}
