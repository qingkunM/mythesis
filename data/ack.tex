
%%% Local Variables:
%%% mode: latex
%%% TeX-master: "../main"
%%% End:

\begin{ack}

回顾自己五年博士求学路,万般滋味涌上心头。求学期间成长了很多,从当初被动地接受知识,到独立思考主动学习;从遇事懵懵懂懂不知所措,到沉着冷静地分析解决问题;从待人处事羞赧畏缩,到落落大方和谐相处。这不仅仅是知识的增长、思维方式的成长,更是全面素质的提高。谨借此机会向博士期间给予我的老师、同学亲友致以衷心的感谢!

衷心感谢我的导师沈荣骏院士。沈院士虽然因为工作、身体原因,不能时常在身边指导,但他仍然心系学生,经常和我们在电话中交流,每次回长沙都会召集我们听取课题进展汇报。每次交流都会给予我中肯的建议,使我受益匪浅!

衷心感谢我的博士导师唐朝京教授!唐老师在学术上一丝不苟的态度,高瞻远瞩的规划,科学地发现、分析、解决问题的能力等方方面面都给我们树立了标杆式的榜样。唐老师虽然承担了行政职务,但在完成行政工作之余,经常看到唐老师研究学术问题到深夜,唐老师在科研方面投入的精力让人惊叹。唐老师为我们建立了定期学术汇报、讨论的机制,以时时跟踪我们的学术思想动态和进展;我的开题是唐老师集合众老师与我正式讨论了四、五次的结果。唐老师在我的课题研究过程中,起了重大的引导、监督、督促、纠正的作用。每次课题研究遇到困境,和唐老师的讨论交流总能激发新的思路;每当对自己的研究方向没有信心或懈怠时,唐老师总能给予鼓励或当头棒喝,让我从新起航;在我论文成型时,唐老师又总是不辞辛苦的一遍又一遍的修改。唐老师严谨、耐心的治学态度深深的影响了我,唐老师的帮助是课题研究成功的必要条件。在此,再次向唐老师表达我最衷心的感谢和崇高的敬意!
%  衷心感谢我的博士导师唐朝京教授!我从 2013 年受教于唐老师就读博士研究
%  生至今,他严肃认真的治学态度、深厚的学术功底、严密的思维、渊博的知识和
%  分析问题、提炼问题的能力,使我在学业上受益匪浅。我在博士论文选题、研究
%  和论文写作的整个过程,都是在唐老师的悉心指导和严格要求下完成的。在几年
%  的学习过程中,不仅使我的学术水平得到了很大的提高,在学术道德方面,我也
%  有了更加深刻的认识,对自己的要求也更为严格。唐老师极富责任感、对待科研
%  非常严谨,这种品质深深地影响着我,促使我形成严谨求实的学术作风,这将成
%  为我人生的一大笔财富。唐老师还对我的论文写作方面进行了细心的指导,无私
%  地传授给我论文写作的大量规律和经验。唐老师对工作认真勤恳的态度以及在学
%  术上不断进取的精神永远是我今后工作学习的榜样!在此谨向他表示最衷心的感
%  谢和最诚挚的敬意!
  
衷心感谢张权副教授!是您带领我走进了信息安全研究的科研殿堂。感谢王剑教授、刘俭老师、张琛老师、张磊副教授、冯超老师、李瑞林老师等诸位老师对我的指导和关心!
  
感谢李孟君师兄、吴波师兄、苏云飞师兄、解纬师兄、张博师兄、王少磊师兄、帅博师兄、李海峰师兄,感谢王强、温沙蒙、毕兴、刘毅、张斌、陈夏阳、叶嘉曦、张兴、冯冈夫、黄安琪,和你们思想碰撞的每一个火花都是我课题研究的强大助力!

感谢我从硕士一直陪伴我的好兄弟、王青平、张可、杨炯、王丁和、张军、马跃、程龙旺、谢启友、顾承飞、王亚森!感谢我的三任室友李华、秦立龙、马佳智!

感谢我的父母、岳父岳母以及所有的亲人!

感谢我的爱人李薇,相处的九年时光里,是你的陪伴和鼓励让我度过了一段段艰难的时光!
  
谨以此文献给所有关心我支持我的亲人和朋友们!

\end{ack}
