\begin{resume}

  \section*{发表的学术论文} % 发表的和录用的合在一起
%公开版
%  \begin{enumerate}[{[}1{]}]
%  \addtolength{\itemsep}{-.36\baselineskip}%缩小条目之间的间距,下面类似
%  \item Qingkun M, Chao F, Bin Z, Chaojing T. Assisting in Auditing of Buffer Overflow Vulnerabilities via Machine Learning, Mathematical Problems in Engineering, 2017.(SCI检索,检索号:00041904420001)
%  
%  \item Qingkun M, Shameng W, Bin Z, Chaojing T. Automatically Discover Program Vulnerability Through Similar Functions. PIERS2016.(EI检索,检索号20165203169818)
%  
%  \item Qingkun M, Bin Z, Chao F, Chaojing T. Detecting Buffer Boundary Violations based on SVM. ICISCE2016. (EI检索,检索号:20165003106907)
%  
%  \item Qingkun M, Shameng W, Chao F, Chaojing T. Predicting buffer overflow using semi-supervised learning. CISP-BMEI2016.(EI检索,检索号:20171303496365)
%  
%  \item Qingkun M, Shameng W, Chao F, Chaojing T. Predicting Integer Overflow through Static Integer Operation Attributes. ICCSNT2016. (EI检索,检索号:20180104610722)
%  
%  \item Shameng W, Qingkun M, Chao F, Chaojing T. Protocol Vulnerability Detection Based on Network Traffic Analysis and Binary Reverse Engineering. PLOS ONE, 2017. (SCI检索 000413195900048)
%    
%  \item Shameng W, Qingkun M, Chao F, Chaojing T. A Model-Guided Symbolic Execu-
%  tion Approach for Network Protocol Implementations and Vulnerability Detection.
%  PLOS ONE, 2017. (SCI检索,检索号:000413195900048)
%  
%  \item Shameng W, Chao F, Qingkun M, Bin Z, Ligeng W, Chaojing T. Testing Network
%  Protocol Binary Software with Selective Symbolic Execution. CIS2016. (EI检索,检索号:20171203453840)
%  
%  \item Shameng W, Chao F, Qingkun M, Bin Z, Ligeng W, Chaojing T. Analyzing Net-
%  work Protocol Binary Software with Joint Symbolic Execution. ICSAI2016. (EI检索,检索号:20171203462505)
%  
%  \item Shameng W, Chao F, Qingkun M, Bin Z, Ligeng W, Chaojing T. Model-Guided
%  Symbolic Execution Testing for Network Protocol Binary Software. PIC2016. (EI
%  检索,检索号:20173003971908)
%  
%  \item Shameng W, Chao F, Qingkun M, Bin Z, Ligeng W, Chaojing T. Multi-Packet
%  Symbolic Execution Testing for Network Protocol Binary Software. ICCSNT2016.
%  (EI检索,检索号:20180104611161)
  
%  \item 
  
%  \end{enumerate}
  %盲评版
  \begin{enumerate}[{[}1{]}]
  %  \addtolength{\itemsep}{-.36\baselineskip}%缩小条目之间的间距,下面类似
    \item 第一作者. Assisting in Auditing of Buffer Overflow Vulnerabilities via Machine Learning, Mathematical Problems in Engineering, 2017.(SCI检索,检索号:00041904420001)
    
    \item 第一作者. Automatically Discover Program Vulnerability Through Similar Functions. PIERS2016.(EI检索,检索号20165203169818)
    
    \item 第一作者. Detecting Buffer Boundary Violations based on SVM. ICISCE2016. (EI检索,检索号:20165003106907)
    
    \item 第一作者. Predicting buffer overflow using semi-supervised learning. CISP-BMEI2016.(EI检索,检索号:20171303496365)
    
    \item 第一作者. Predicting Integer Overflow through Static Integer Operation Attributes. ICCSNT2016. (EI检索,检索号:20180104610722)
    
    \item 第二作者. Protocol Vulnerability Detection Based on Network Traffic Analysis and Binary Reverse Engineering. PLOS ONE, 2017. (SCI检索 000413195900048)
      
    \item 第二作者. A Model-Guided Symbolic Execu-
    tion Approach for Network Protocol Implementations and Vulnerability Detection.
    PLOS ONE, 2017. (SCI检索,检索号:000413195900048)
    
    \item 第三作者. Testing Network
    Protocol Binary Software with Selective Symbolic Execution. CIS2016. (EI检索,检索号:20171203453840)
    
    \item 第三作者. Analyzing Net-
    work Protocol Binary Software with Joint Symbolic Execution. ICSAI2016. (EI检索,检索号:20171203462505)
    
    \item 第三作者. Model-Guided
    Symbolic Execution Testing for Network Protocol Binary Software. PIC2016. (EI
    检索,检索号:20173003971908)
    
    \item 第三作者. Multi-Packet
    Symbolic Execution Testing for Network Protocol Binary Software. ICCSNT2016.
    (EI检索,检索号:20180104611161)
    
  %  \item 
    
    \end{enumerate}

  \section*{参与的主要科研项目} % 有就写,没有就删除
  \begin{enumerate}[{[}1{]}]
  \addtolength{\itemsep}{-.36\baselineskip}%
  \item XXXX信息系统,海军项目,项目主要负责人.
  \item 全军XXXX装备业务系统,总参项目,项目主要负责人.
  \item XXXX挖掘验证测试技术,国家863项目,项目主要参与者.
  \item 大规模XXXX,2016国家重点研发计划,项目主要参与者.
  \end{enumerate}
\end{resume}
