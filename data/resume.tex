\begin{resume}

  \section*{发表的学术论文} % 发表的和录用的合在一起

  \begin{enumerate}[{[}1{]}]
%  \addtolength{\itemsep}{-.36\baselineskip}%缩小条目之间的间距,下面类似
  \item Meng Q K, Chao F, Zhang B, et al. Assisting in Auditing of Buffer Overflow Vulnerabilities via Machine Learning. In press.(已被Mathematical Problems in Engineering录用,SCI源刊.)
  
  \item Meng Q K, Wen S M, Zhang B, et al. Meng Q K, Chao F, Zhang B, et al.  Automatically Discover Program Vulnerability Through Similar Functions. 2016 37th Progress In Electromagnetics Research Symposium.(EI收录,检索号20165203169818)
  
  \item Meng Q K, Zhang B, Feng C, et al. Detecting Buffer Boundary Violations based on SVM. 2016 3rd International Conference on Information Science and Control Engineering. (EI收录,检索号:20165003106907)
  
  \item Meng Q K, Wen S M, Feng C, et al. Predicting buffer overflow using semi-supervised
  learning. 2016 9th International Congress on Image and Signal Processing, BioMedical Engineering and Informatics. (EI检索,检索号:20171303496365)
  
  \item Meng Q K, Wen S M, Feng C, et al. Predicting Integer Overflow through Static Integer Operation Attributes. 2016 5th International Conference on Computer Science and Network Technology. (IEEE国际会议,已入ieee Xplore)
  
  \end{enumerate}

%  \section*{研究成果} % 有就写,没有就删除
%  \begin{enumerate}[{[}1{]}]
%  \addtolength{\itemsep}{-.36\baselineskip}%
%  \item 任天令, 杨轶, 朱一平, 等. 硅基铁电微声学传感器畴极化区域控制和电极连接的
%    方法: 中国, CN1602118A. (中国专利公开号.)
%  \item Ren T L, Yang Y, Zhu Y P, et al. Piezoelectric micro acoustic sensor
%    based on ferroelectric materials: USA, No.11/215, 102. (美国发明专利申请号.)
%  \end{enumerate}
\end{resume}
