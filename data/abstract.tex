\begin{cabstract}
%wb
随着信息技术的发展,软件已成为与世界经济、文化、科技、教育和军事发展息息相关的重要元素,软件作为信息系统中的核心基础设施之一,广泛应用于通信、金融、医疗等众多领域。但随着软件功能的日益增多和规模的日益庞大,软件存在故障缺陷和安全漏洞不可避免,从而给信息系统的可靠性和安全性带来严重的威胁。近年来,随着国内外各种开源社区的兴起,开源代码的使用越来越广泛。源代码软件的漏洞挖掘技术能够针对性的开源软件进行挖掘,掌握开源软件的漏洞挖掘技术对我国、我军的信息安全具有重大战略意义。
%wb

论文以源代码软件为漏洞挖掘目标,研究了多种源代码动、静态分析技术,其主要工作和创新如下:

针对多种源代码漏洞不能通过单一的中间表示检测的问题,论文提出了一种基于程序性质图的源代码软件漏洞挖掘方法。首先利用语法解析器解析源代码,并依次生成语法分析树、抽象语法树、控制流图、数据流图;然后利用性质图聚合抽象语法树、控制流图以及数据流图形成程序性质图并定义程序性质图的基本遍历方法;最后,组合程序性质图遍历方法检测缓冲区溢出漏洞、格式化字符串漏洞以及Use After Free漏洞。实验结果表明,该方法能有效的检测各种类型的源代码漏洞。

针对利用基于程序性质图挖掘缓冲区溢出漏洞漏报率高的问题,提出了一种基于机器学习的缓冲区溢出漏洞预测方法。该方法首先总结了7类缓冲区溢出漏洞静态特征,分别为sink类型、缓冲区位置、容器、索引/地址/长度复杂度、边界检测、循环/条件/函数调用深度以及是否输入可控;然后,通过扩展的程序性质图检测缓冲区溢出漏洞的各类性质并将其向量化;然后利用有监督机器学习算法在已标记的训练集上训练分类器;最后利用此分类器在新的源代码程序中预测缓冲区溢出漏洞。实验结果表明,该方法能在较低误报的情况下预测漏洞。

针对循环程序导致的符号执行路径指数级爆炸问题,提出了一种新的结合静态程序分析的高效符号执行技术。区别于其他符号执行技术,该技术通过静态程序分析方法,从程序的控制流图中计算出循环程序的不变式;然后对程序进行插桩用循环不变式代替循环,形成新的控制流图;最后在新的控制流图上进行符号执行。对比实验表明,该方法能有效的提高符号执行的效率。

针对现有导向模糊测试方法不能细粒度导向的问题,提出了一种细粒度变异的导向模糊测试方法。该方法首先利用导向模糊测试收集测试用例;然后利用时间递归神经网络训练出一个模型,用于判断对靠近目标区域起关键作用的字段,同时收集每个字段的权重;最后,通过上述模型判断当前测试用例的关键字段,并利用关键字段权重进行细粒度的变异生成测试用例。实验结果表明,相对于导向模糊测试以及一般的模糊测试,该方法能更有效的导向目标区域并发现漏洞。


\end{cabstract}
\ckeywords{漏洞挖掘;软件安全;符号执行;导向模糊测试;测试用例生成}

\begin{eabstract}
With the development of information technology, software has become an important
element of the world economy, culture, science \&technology, education and military de-
velopment. As one of the core infrastructures of information system, software is widely
used in communication, finance, medicine, etc. However, with the increasing software
function and the increasing scale, the software defects and security vulnerabilities are un-
avoidable, which brings serious threat to the reliability and security of the information
system. In the military field, the information and communication system represented by
C 4 ISR is the nerve center of the modern army, which plays a vital role in winning the
information war under modern conditions. The vulnerabilities, if mastered by the enemy,
will cause incalculable damage in wartime. At present, many countries are competing
to develop the field of cyberspace, the United States, Russia, Japan and other countries
treat the cyberspace as the fifth dimension of war space after the land, sea, air and space.
One of the key elements of cyberspace attack and defense is vulnerability, vulnerability
detection and exploitation for both sides of offense and defense has important value, the
appearance of the Stuxnet virus indicates that the software vulnerabilities have been ap-
plied to cyberspace operations. It is of great practical significance to study the techniques
of vulnerability detection and discover the vulnerabilities in software.

After many years of research, software vulnerability detection techniques have been
put forward, such as static analysis, fuzzing test, symbolic execution and so on. But each
technique has different advantages and disadvantages. The static analysis has high cover-
age and can detect many kinds of vulnerabilities, but its false positive rate is high. Fuzzing
has low false positive rate but low code coverage and low efficiency, a large number of
test cases may repeat the same program path. Symbolic execution is considered to be
the most promising technique in vulnerability detection. The symbolic execution is path
sensitive, so the false positive rate is low, and it can theoretically traverse all the paths of
the program with constraint solver and once each path, the code coverage and execution
efficiency is high, but the biggest problem that plagued symbolic execution is the path
explosion problem.

This thesis analyzes the main problems of current software vulnerability detection,and proposes a new framework of vulnerability detection technique targeting binary pro-
gram, which uses the static analysis to locate suspected vulnerabilities and post-dynamic
testing to verify the suspected vulnerabilities. The combination of program analysis and
testing techniques can learn from each other and improve the efficiency of vulnerabil-
ity detection. Based on the translation of the binary code to the intermediate code, the
thesis locates the vulnerabilities on the intermediate code through the dataflow analysis
and abstract interpretation, and then verifies the suspected vulnerabilities through program
slicing, backward symbolic execution and guided fuzzing test.

The main work and innovations are as follows:
This thesis presents a method to exploit data flow analysis and abstract interpretation
to locate the vulnerabilities in intermediate code. Based on LLVM IR, this thesis presents
a method based on boundary-based vulnerability model, which exploits data flow analysis
and abstraction to perform vulnerabilities location and screening, and the located vulner-
abilities can direct the program slicing, symbolic execution and fuzzing test. With the
translation from binary code to intermediate code, on the hand it avoids the direct analysis
of the complexity of the binary assembly code, on the other hand it ensures the universality
of the static analysis algorithm, that is, any platform code can be converted to intermedi-
ate code and then use existing algorithms to analyze, to avoid re-design algorithm for
each platform. Based on the theory of data flow analysis, this thesis studies the algorithm
of detecting UAF and double free vulnerabilities. Based on the theory of abstract inter-
pretation, this thesis studies the method of interval analysis of key variables of memory
vulnerabilities.


\end{eabstract}
\ekeywords{ulnerability Detection, LLVM IR, Dataflow Analysis, Abstract In-
terpretation, Program Slicing, Path Merging, Backward Symbolic Execution, Guided
Fuzzing}

