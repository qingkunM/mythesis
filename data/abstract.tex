\begin{cabstract}
%syf
随着信息技术的发展,软件已成为与世界经济、文化、科技、教育和军事发展息息相关的重要元素,广泛应用于通信、金融、医疗等众多领域。 %syf
无论是商业软件还是程序员自行开发的小程序,开源代码/组件的使用已经变得越来越普遍,开源已经成为了一种趋势。源代码软件漏洞的影响越来越大,基于开源软件漏洞的网络攻击活动数量在逐年增长。源代码软件漏洞挖掘技术能够针对性的对开源软件进行挖掘,掌握开源软件的漏洞挖掘技术对我国、我军的信息安全具有重大战略意义。

论文围绕源代码软件漏洞挖掘中的关键技术展开研究。%wsm
现有的源代码软件漏洞挖掘方法还存在一些关键问题:在静态分析方面,现有方法存在支持的漏洞类型少、挖掘精度低的问题;在动态测试方面,符号执行和模糊测试技术虽然都能挖掘漏洞,但是符号执行存在路径爆炸问题,模糊测试存在覆盖率低、不具备导向性等问题。据此,结合源代码的直接或者间接信息、形式化方法,
论文在漏洞静态分析、符号执行以及模糊测试方面展开了研究,主要工作和创新如下:
%在漏洞挖掘方面存在多种类源代码漏洞挖掘不够精确、软件测试效率低、导向能力差等问题。
%wb
%随着信息技术的发展,软件已成为与世界经济、文化、科技、教育和军事发展息息相关的重要元素,软件作为信息系统中的核心基础设施之一,广泛应用于通信、金融、医疗等众多领域。但随着软件功能的日益增多和规模的日益庞大,软件存在故障缺陷和安全漏洞不可避免,从而给信息系统的可靠性和安全性带来严重的威胁。近年来,随着国内外各种开源社区的兴起,开源代码的使用越来越广泛。源代码软件的漏洞挖掘技术能够针对性的开源软件进行挖掘,掌握开源软件的漏洞挖掘技术对我国、我军的信息安全具有重大战略意义。
%wb

%论文以源代码软件为漏洞挖掘目标,研究了多种源代码动、静态分析技术,其主要工作和创新如下:

针对多种类源代码软件漏洞静态挖掘问题,论文提出了一种基于程序性质图的源代码软件漏洞挖掘方法。首先利用语法解析器解析源代码,依次生成语法分析树、抽象语法树、控制流图、数据流图;然后聚合抽象语法树、控制流图以及数据流图形成程序性质图,并定义程序性质图的基本遍历方式;最后,根据多种源代码漏洞的描述,在组合程序性质图遍历方式的基础上挖掘
%缓冲区溢出漏洞、格式化字符串漏洞以及Use After Free
漏洞。实验结果表明,该方法能有效的检测各种类型的源代码漏洞。

针对缓冲区溢出漏洞挖掘精度问题,
论文提出了一种基于机器学习的缓冲区溢出漏洞挖掘方法。该方法首先总结了7类缓冲区溢出漏洞静态特征,分别为sink类型、缓冲区位置、容器、索引/地址/长度复杂度、边界检测、循环/条件/函数调用深度以及是否输入可控;然后,通过扩展的程序性质图检测缓冲区溢出漏洞的各类性质并将其向量化;然后利用有监督机器学习算法在已标记的训练集上训练分类器;最后利用此分类器在新的源代码程序中挖掘缓冲区溢出漏洞。实验结果表明,相对于其他静态分析工具,该方法能在较低误报的情况下挖掘漏洞。

针对符号执行循环程序引起的路径爆炸问题,
论文提出了一种结合静态程序分析的高效符号执行技术。首先,该技术通过静态程序分析方法,从程序的控制流图中计算出循环程序的不变式;然后,对程序进行插桩,用循环不变式代替循环形成新的控制流图;最后,在新的控制流图上进行符号执行。对比实验表明,该方法比单纯的符号执行以及对循环进行定长展开的符号执行发现更多的漏洞,并且耗费的时间更少。

%能有效的提高符号执行的效率。
针对模糊测试导向问题,
论文提出了一种细粒度变异的导向模糊测试方法。该方法首先利用导向模糊测试收集测试用例;然后利用时间递归神经网络训练出一个模型,用于判断对靠近目标区域起关键作用的字段,同时收集每个字段的权重;最后,通过上述模型判断当前测试用例的关键字段,并利用关键字段权重进行细粒度的变异生成测试用例。实验结果表明,相对于导向模糊测试以及普通的模糊测试,该方法能更有效的导向目标区域并发现漏洞。


\end{cabstract}
\ckeywords{漏洞挖掘;软件安全;符号执行;导向模糊测试;测试用例生成}

\begin{eabstract}

With the development of information technology, software has become an important
element of the world economy, culture, technology study, education and military de-
velopment and is widely
used in communication, finance, medicine, etc. No matter in commercial software or small program developed by individual programmers, the use of source code/components has become increasingly common and open souce has become a trend. The vulnerability of source code software is increasing, and the number of network attacks based on open source software vulnerabilities is increasing year by year. The vulnerability mining technology of source code software can be targeted to open source software, and it is of great strategic significance for China and our army to grasp the vulnerability mining technology of open source software.

This paper focuses on the key technologies in source software vulnerability mining. It is found that existing source code vulnerability mining method is not perfect. In the aspect of static analysis, the existing method has the problem of low type of support and low mining accuracy. In terms of dynamic testing, symbolic execution and fuzzy test technology, although can dig holes, but symbolic execution path explosion problems, fuzzy test problems such as low coverage, do not have guidance. Based on the direct or indirect information and formal methods of source code, the paper studies the static analysis, symbol execution and fuzzy test.


The main work and innovation are as follows:

Aiming at solving the problem of statically mining multiple source code software vulnerabilities, a method of source code software vulnerability mining based on code property graph is proposed. At first, parse the source code with the grammar parser to generate the parse tree, abstract syntax tree, control flow graph and data flow graph in turn. Then the abstract syntax tree, control flow diagram, and data flow chart are used to form the code property graph, and the basic traversal method of the program nature graph is defined. Finally, based on the description of multiple source code vulnerability, the flaw is excavated on the basis of the traversal mode of the combination program. Experimental results show that this method can effectively detect all kinds of source code vulnerabilities.

Aiming at solving the problem of mining precision of buffer overflow vulnerability, a method of mining buffer overflow vulnerability based on machine learning is proposed. This method firstly summarizes 7 kinds of static attributes of buffer overflow vulnerabilities, namely sink type, containers, index/address/length, complexity, sanitization, loop/conditions/call depth and input control. Then, the static attributes of the buffer overflow vulnerability are extracted by the extended code property graph and are converted to digital vector. Then, the supervised machine learning algorithms are used to train classifiers on the labeled training data set. Finally, this classifier is used to exploit the buffer overflow vulnerability in the new source code program. The experimental results show that this method can exploit buffer overflow vulnerabilities in low false positives compared with other static analysis tools.

Aiming at solving the problem of path explosion caused by the symbol execution of program containing loops, a high efficient symbol execution technique combined with static program analysis is proposed. Firstly, the static program analysis is used to calculate the loop invariant from the control flow graph of the program. Then, instrumenting the program with loop invariant to generate the new control flow graph. Finally, symbol execution is performed on the new control flow diagram to detect vulnerabilities. The comparison experiment shows that this method can find more vulnerabilities than the normal symbol execution and the specific symbolic execution which unrolls loops with fixed time and it takes less time. 

Aiming at solving the problems of directed fuzzing, a method of fine-grained variation directed fuzzing is proposed. Firstly, the method uses directed fuzzing to collect test cases. Then, Long Short-Term Memory is used to train a model to determine the which fields that play a key role in testing the target area; in the meantime the weight of each field is collected. Finally, the key field of the current test case is dynamically generated by the above model, and the key field weights are used to generate the test cases with fine-grained variation. The experimental results show that the method is more effective in guiding the target area and detecting vunerabilities than the directed fuzzing and the normal fuzzing.


\end{eabstract}
\ekeywords{Vulnerability Detection, Software Security, Symbolic Execution, Guided Fuzzing, Testcase Generation}

