\chapter{总结与展望}

\section{论文主要工作总结}
%syf
论文以源代码程序为漏洞挖掘目标,分析了当前软件漏洞挖掘面对的主要问
题,采用先静态分析定位疑似漏洞,后动态
测试验证疑似漏洞的漏洞挖掘思路,通过把程序分析与测试技术相结合,取长补
短,提高漏洞挖掘的效率。一方面由于静态分析代码覆盖率高,能够获取代码的
全局信息,因而能够描述的漏洞类型多,漏洞挖掘的漏报率低,另一方面,可以
通过动态测试的方法验证静态分析发现的漏洞是否为真实漏洞,去除误报,同时,
采用先静态定位后动态验证的方法使得动态测试的针对性更强,减少了对整个程
序盲目测试所带来的时间和资源开销。
%syf

论文研究了多种源代码动、静态分析技术,其主要工作和创新如下:

(1)提出了一种基于程序性质图的源代码软件漏洞挖掘方法,用于解决多种源代码漏洞不能通过单一的中间表示检测的问。首先利用语法解析器鲁棒的解析源代码,并依次生成语法分析树、抽象语法树、控制流图、数据流图;然后利用性质图聚合抽象语法树、控制流图以及数据流图形成程序性质图,并且定义程序性质图的基本遍历方法;最后,组合程序性质图遍历方法检测缓冲区溢出漏洞、格式化字符串漏洞以及Use After Free漏洞。该方法能有效的检测各种类型的源代码漏洞。

(2)提出了一种基于机器学习的缓冲区溢出漏洞挖掘方法。该方法首先总结了7类缓冲区溢出漏洞静态特征,分别为sink类型、缓冲区位置、容器、索引/地址/长度复杂度、边界检测、循环/条件/函数调用深度以及是否输入可控;然后,通过扩展的程序性质图检测缓冲区溢出漏洞的各类性质并将其向量化;然后利用有监督机器学习算法在已标记的训练集上训练分类器;最后利用此分类器在新的源代码程序中挖掘缓冲区溢出漏洞。该方法以较低的误报误报率情况下挖掘漏洞。

(3)提出了一种新的结合静态程序分析的高效符号执行技术,用于缓解循环程序导致的符号执行路径指数级爆炸问题。区别于其他符号执行技术,该技术首先通过静态程序分析方法,从程序的控制流图中计算出循环程序的不变式;然后对程序进行插桩用循环不变式代替循环,形成新的控制流图;最后在新的控制流图上进行符号执行。该方法能有效的提高符号执行的效率。

(4)提出了一种细粒度变异的导向模糊测试方法。该方法首先利用AFL-go收集测试用例;然后利用时间递归神经网络训练出一个模型,用于判断对靠近目标区域起关键作用的字段,同时收集每个字段的权重;最后,通过上述模型判断当前测试用例的关键字段,并利用关键字段权重进行细粒度的变异生成测试用例。实验结果表明,相对于AFL-go以及AFL,该方法能更有效的导向目标区域并发现漏洞。


\section{工作展望}
%wb
由于学术水平和精
力所限,本文的研究成果还有许多进一步拓展的空间。另一方面,理论的发展技
术的进步不会停止,这也会引导研究者去寻找新的更好的漏洞挖掘方法。因此,
未来一段时期内需要进一步研究的主要内容包括:
%wb

(1)论文第三章中研究了基于有监督学习的缓冲区溢出漏洞预测方法,
其静态特征需要人为指定,训练数据需要人为标定。特征的指定需要人的经验,训练数据的
标定也是一件非常耗时的工作。所以探索利用深度学习的方法去自动学习漏洞特征是下一步研究方向。

(2)论文第四章中研究了结合静态程序分析的高效符号执行技术,运用抽象解释的方法求解循环程序的不变式。
虽然能大大增加符号执行的效率,但同时降低了符号执行的精度,会导致产生虚假的测试用例。如何深层次
的结合抽象解释和符号执行技术,在精度和效率之间达到一个更好的平衡是下一步研究的方向。此外,在
已知目标区域的情况下如何进行导向符号执行是另外一个需要研究的工作。

(3)论文第五章研究了基于细粒度变异的导向模糊测试技术,但是其训练和动态测试是分离的两部分,动态测试的
结果不能及时的反馈给训练过程做实时修正,如何将训练过程与动态测试过程联动起来增加测试的效率和效果
是下一步的研究方向。
